\chapter{Music Analysis Techniques} 
\label{Chapter2} 

\lhead{Chapter 2. \emph{Music Analysis Techniques}} 
The main subject of MIR regards the \textit{extraction and inference of musically meaningful features, indexing of music} (through these features) and the development of \textit{search and retrieval schemes} \cite{downieMIR}. In other terms, the main target of MIR is to make all the music over the world easily accessible to the user \cite{downieMIR}. During the last two decades, several approaches have been developed, which mainly differ in the music perception category of the features they deal with. These categories generally are: \textit{music content}, \textit{music context}, \textit{user properties} and \textit{user context} \cite{gomez14}. \textit{Music content} deal with aspects that are directly inferred by the audio signal (such as melody, rhythmic structure, timbre) while \textit{music context} refers to aspects that are not directly extracted from the signal but are strictly related to it (for example label\cite{pachet00}, artist and genre information \cite{perfe11} \cite{aizenberg12}, year of release \cite{vangulik05}, lyrics \cite{coelho13} and semantic labels). Regarding the user, the difference between \textit{user context} and \textit{user properties} lies on the stability of aspects of the user himself. The former deals with aspects that are subject to frequent changes (such as mood or social context), while the latter refers to aspects that may be considered constant or slowly changing, for instance his music taste or education \cite{gomez14}. \\In this chapter, we will focus on the differences between the categories \textit{music content} and \textit{music context}. 


\section{Metadata}
By metadata we mean all the descriptors about a track that are not based on the \textit{music context}. Therefore, they are not directly extracted from the audio signal but rather from external sources. They began to be deeply studied since the early 2000s, when first doubts about an upper threshold of the performance of audio content analysis systems arised \cite{aucou04}. Researchers then started exploring the possibility of performing retrieving tasks on written data that is related to the artist or to the piece. \\At first, the techniques were adapted from the Text-IR ones, but it was immediately clear that retrieving music is fairly more complex than retrieving text, because the music retrieved should also satisfy the musical taste of the user who performed the query. 
\\The techniques used in this category may differ both in the sources used for retrieving data and in the way of computing a similarity score, and clearly the performance of a system using metadata for similarity computation is highly affected by both of these factors. Sources may include \cite{bogdanov13}:
\begin{itemize}
\item manual annotation: description provided by experts; they may be referred to genre, mood, instrumentation, artist relations.
\item collaborative filtering data: data indirectly provided by users of web communities, in the form of user ratings or listening behaviour information.
\item social tags: data directly provided by users of social network of music (such as \textit{Last.fm}\footnote{\url{http://last.fm}}) or social games.
\item information automatically mined from the Web. Sources in these cases may include web-pages related to music or microblogs (for instance the very popular \texit{Twitter}).
\end{itemize}
 The availability of some of them greatly depends on the size of the music collection under consideration; for instance, as manual expert annotations might be very accurate, they would be extremely costly and probably infeasible on large collections \cite{Szyma09}. In contrast, collaborative filtering data may be the most studied technique, given that it may be applied to other different fields (such as movies or books recommendation) with just little changes. Sources are picked also in relation to the subject of the research or of the system, that may be for example a recommendation or a similarity computation system. At this point, it's important to highlight the difference between the two of them: a recommendation system not only has to find similar music, but has also to take into account the personal taste of the user, and therefore it's generally considered more complex \cite{bogdanov13}. For this kind of systems, collaborative filtering data has shown to lead to better results \cite{green09}. However, in the field of music similarity computation, social tags and keywords extracted from webpages have shown good performances. The computation of similarity may happen through a Vector Space Model (a technique adapted from the Text-IR) or through co-occurence analysis. In the next subsections we will see the characteristics and the performance of these two techniques.

\subsection{Computing Similarity With a Vector Space Model} 
\subsection{Computing Similarity With Co-Occurence Analysis}

\section{Audio Content Analysis}
The main idea behind this kind of analysis is to directly extract useful information, through some algorithms (or library of algorithms), from the audio signal itself. The type of content information extracted may greatly vary in relation to the need of the research, but we can mainly distinguish four categories \cite{bogdanov13}:
\begin{itemize}
\item \textit{timbral} information: related to the overall quality and color of the sound.
\item \textit{temporal} information: related to rhythmic aspects of the composition, such as tempo or length of measures.
\item \textit{tonal} information: directly linked to the frequency analysis of the signal and to the pitch. It can describe \texit{what notes are being played} or the tonality of a given track.
\item \textit{inferred semantic} information: information inferred (usually through machine learning techniques) from the previous categories, in the attempt of giving a more defined and understable shape to the data collected. This kind of information may include descriptors such as genre, valence or arousal.
\end{itemize}

Information extracted through this family of techniques may also be categorized in the following way:
\begin{itemize}
\item Low-level data: information that has no musical meaning and that, more in general, is not interpretable by humans. Examples of this kind of descriptors are Mel Frequency Cepstral Coefficients (MFCCs) and Zero Crossing Rate (ZCR).
\item Mid-level data: information that has musical meaning but that is related to low-level music features. This kind of category mainly includes temporal and tonal descriptors. 
\item High-level data: corresponding to inferred semantic information.
\end{itemize}

Many of the studies conducted on the computation of music similarity through audio content descriptors have solely focused on low-level and timbral information, because this has been proved to bring alone to acceptable results with proper similarity measures \cite{mirage07}. However, more recent studies have shown some evidence of advantages in using high-level descriptors \cite{barrington07} \cite{west07} and, more in general, the most performant systems use data from all of these categories. \\In the next sections, a more detailed look among most important descriptors will be given.

\subsection{Low-level Data}

\subsection{Mid-level Data}
\subsection{High-level Data}

\subsection{Main Tools For Extracting Audio Content}
\subsubsection*{Essentia}
\subsubsection*{Echonest}
\subsubsection*{jMIR}
\subsubsection*{MIRtoolbox}

\section{Conceptual Differences Between Metadata and Audio Content Information}
The performance of content-based approaches is considerably lower \cite{slaney2011}.
It is challenging to try to make the so-called \textit{semantic gap} smaller \cite{aucou2009}

The advantage of relying on the audio signal over, say, expert annotations, is that
the process is objective and can be automated to a large extent. However, extracting
the features can be computationally costly \cite{schma13}. Another
limitation is that there might be features like the release date, the “freshness,” or
popularity of a track, which can be relevant in the playlist generation process but that
cannot be extracted from the audio signal \cite{celma08}.

When used in an automated process, data completeness and consistency are crucial.
Another potential problem is that not all types of metadata are objective, and annotations regarding, for example, the mood or the genre of a track can be imprecise or inconsistent \cite{celma2010}.

(speaking of tags) Although such annotations can be rich and diverse, the perception of music is again subjective and can even be influenced by the perception of other people \cite{mcdermott12}; tags only for popular songs \cite{celma2010}

When dealing with track ratings: grabbed from a wall posting on Facebook \cite{germain13} or a tweet on twitter \cite{hauger13}, 1-to-5 rating scales like on iTunes. Challenges: problem of data sparsity (especially for the tracks from the long tail), a positivity bias (the phenomenon that most of the ratings are highly positive and negative feedback is rare \cite{celma2010}).